\documentclass{article}

\usepackage{fullpage}
\usepackage{graphicx}
\usepackage{natbib}

\begin{document}

\section{Introduction}

There are several methods for computing error bands around estimated
breakpoint locations \citep{Rigaill, Luong}.

\citet{Erdman} proposed a Bayesian change-point method but it can not
be used to compute confidence bands.

\section{Methods}

The goal of this paper is to quantitatively compare these models in
real data sets with labels on pairs of 1breakpoint regions: ``this
breakpoint obviously should have a larger error band than this other
one.'' 

\includegraphics[width=\textwidth]{figure-labels-5}

Using these labels we can compute an error function for any model (the
number of incorrect labels).

In models.R I tried to compute models using postCP and EBS packages,
but ran into some problems:
\begin{itemize}
\item postCP: matrix of zeros for confidence intervals.
\item EBS: too slow for $N>10000$ data points to
  segment. \citet{Rigaill} claims $O(KN^2)$ time complexity where $K$
  is the maximum number of segments. QUESTION: what about memory?
\end{itemize}
QUESTION: is there any existing method with an implementation that
works in practice for $N>10000$ data points?

QUESTION: do these methods have a parameter that needs to be learned?
Train/test split?

\bibliographystyle{abbrvnat}
\bibliography{refs}

\end{document}
